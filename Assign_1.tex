
\let\negmedspace\undefined
\let\negthickspace\undefined
\documentclass[journal,12pt,twocolumn]{IEEEtran}
\usepackage{cite}
\usepackage{amsmath,amssymb,amsfonts,amsthm}
\usepackage{algorithmic}
\usepackage{graphicx}
\usepackage{textcomp}
\usepackage{xcolor}
\usepackage{txfonts}
\usepackage{listings}
\usepackage{enumitem}
\usepackage{mathtools}
\usepackage{gensymb}
\usepackage[breaklinks=true]{hyperref}
\usepackage{tkz-euclide} % loads  TikZ and tkz-base
\usepackage{listings}

\DeclareMathOperator*{\Res}{Res}
%\renewcommand{\baselinestretch}{2}
\renewcommand\thesection{\arabic{section}}
\renewcommand\thesubsection{\thesection.\arabic{subsection}}
\renewcommand\thesubsubsection{\thesubsection.\arabic{subsubsection}}

\renewcommand\thesectiondis{\arabic{section}}
\renewcommand\thesubsectiondis{\thesectiondis.\arabic{subsection}}
\renewcommand\thesubsubsectiondis{\thesubsectiondis.\arabic{subsubsection}}

% correct bad hyphenation here
\hyphenation{op-tical net-works semi-conduc-tor}
\def\inputGnumericTable{}                                 %%

\lstset{
%language=C,
frame=single, 
breaklines=true,
columns=fullflexible
}
%\lstset{
%language=tex,
%frame=single, 
%breaklines=true
%}

\begin{document}
%


\newtheorem{theorem}{Theorem}[section]
\newtheorem{problem}{Problem}
\newtheorem{proposition}{Proposition}[section]
\newtheorem{lemma}{Lemma}[section]
\newtheorem{corollary}[theorem]{Corollary}
\newtheorem{example}{Example}[section]
\newtheorem{definition}[problem]{Definition}
%\newtheorem{thm}{Theorem}[section] 
%\newtheorem{defn}[thm]{Definition}
%\newtheorem{algorithm}{Algorithm}[section]
%\newtheorem{cor}{Corollary}
\newcommand{\BEQA}{\begin{eqnarray}}
\newcommand{\EEQA}{\end{eqnarray}}
\newcommand{\define}{\stackrel{\triangle}{=}}

\bibliographystyle{IEEEtran}
%\bibliographystyle{ieeetr}


\providecommand{\mbf}{\mathbf}
\providecommand{\pr}[1]{\ensuremath{\Pr\left(#1\right)}}
\providecommand{\qfunc}[1]{\ensuremath{Q\left(#1\right)}}
\providecommand{\sbrak}[1]{\ensuremath{{}\left[#1\right]}}
\providecommand{\lsbrak}[1]{\ensuremath{{}\left[#1\right.}}
\providecommand{\rsbrak}[1]{\ensuremath{{}\left.#1\right]}}
\providecommand{\brak}[1]{\ensuremath{\left(#1\right)}}
\providecommand{\lbrak}[1]{\ensuremath{\left(#1\right.}}
\providecommand{\rbrak}[1]{\ensuremath{\left.#1\right)}}
\providecommand{\cbrak}[1]{\ensuremath{\left\{#1\right\}}}
\providecommand{\lcbrak}[1]{\ensuremath{\left\{#1\right.}}
\providecommand{\rcbrak}[1]{\ensuremath{\left.#1\right\}}}
\theoremstyle{remark}
\newtheorem{rem}{Remark}
\newcommand{\sgn}{\mathop{\mathrm{sgn}}}
\providecommand{\abs}[1]{\left\vert#1\right\vert}
\providecommand{\res}[1]{\Res\displaylimits_{#1}} 
\providecommand{\norm}[1]{\left\lVert#1\right\rVert}
%\providecommand{\norm}[1]{\lVert#1\rVert}
\providecommand{\mtx}[1]{\mathbf{#1}}
\providecommand{\mean}[1]{E\left[ #1 \right]}
\providecommand{\fourier}{\overset{\mathcal{F}}{ \rightleftharpoons}}
%\providecommand{\hilbert}{\overset{\mathcal{H}}{ \rightleftharpoons}}
\providecommand{\system}{\overset{\mathcal{H}}{ \longleftrightarrow}}
	%\newcommand{\solution}[2]{\textbf{Solution:}{#1}}
\newcommand{\solution}{\noindent \textbf{Solution: }}
\newcommand{\cosec}{\,\text{cosec}\,}
\providecommand{\dec}[2]{\ensuremath{\overset{#1}{\underset{#2}{\gtrless}}}}
\newcommand{\myvec}[1]{\ensuremath{\begin{pmatrix}#1\end{pmatrix}}}
\newcommand{\mydet}[1]{\ensuremath{\begin{vmatrix}#1\end{vmatrix}}}

\let\vec\mathbf

\vspace{3cm}

\title{
\textbf{Assignment 1} \\ \large \textbf{AI1110}: Probability and Random Variables 


}
\author{ Rishitha Surineni\\ cs22btech11050} 
	



% make the title area
\maketitle

\newpage

%\tableofcontents

\bigskip

\renewcommand{\thefigure}{\theenumi}
\renewcommand{\thetable}{\theenumi}

\textbf{12.13.1.15: Question:}\\
 	Consider the experiment of throwing a die,if a multiple of 3 comes up,throw the die again and if any other number comes,toss a coin.Find the conditional probability of the event `the coin shows a tail',given that `at least one die shows a 3'.
\\\\
 \textbf{Answer:0.}\\
 \\
 \textbf{Solution:}
 \\
 Given that a die is thrown and if the outcome is a multiple of 3 i.e.,3 or 6 then another die is thrown, else a coin is tossed.
 \\ Let S be the sample space of the experiment then 
\begin{align*}
        S = \{ &(1,H), (1,T), (2,H), (2,T), (4,H), (4,T), \\
                &(5,H), (5,T), (3,1), (3,2), (3,3), (3,4), (3,5),\\
                &(3,6), (6,1), (6,2), (6,3), (6,4), (6,5), (6,6)\}
\end{align*}
Let A be the event that `the coin shows a tail'.
 \\B be the event that `atleast one die shows 3'.
\\ Need to Find, Conditional Probability of the event `the coin shows a tail',given that `at least one die shows a 3', i.e., \pr{A|B}
\begin{align}
    \label{eq:1}
    \pr{A|B}=\frac{\pr{AB}}{\pr{B}}
\end{align}
\begin{align*}
        A=\{&(1,T), (2,T), (4,T), (5,T)\}\\
        B=\{&(3,1), (3,2), (3,3), (3,4), (3,5), (3,6), (6,3)\}
\end{align*}
Let n(E) represents the number of favorable outcomes of the event E.
\\Here, the events A and B are disjoint(as no ordered pair is common to both the events).
\\ Hence, n(AB)=0
\begin{align}
       \pr{AB}=\frac{n(AB)}{n(S)}\\
       \pr{AB}=\frac{0}{20}\\
       \pr{AB}=0
\end{align}
Similarly,
\begin{align}
       \pr{B}=\frac{n(B)}{n(S)}\\
       \pr{B}=\frac{7}{20}
\end{align}
Therfore, \\
From eq(1),eq(4),eq(6)\\
\begin{align}
       \pr{A|B}=\frac{\pr{AB}}{\pr{B}}\\
       \pr{A|B}=\frac{0}{\frac{7}{20}}\\
       \pr{A|B}=0
\end{align}
Hence,
\\Probability of the event `the coin shows a tail',given that `at least one die shows a 3' is 0.
\end {document}